\documentclass[12pt,a4paper]{article}
\usepackage[slovene]{babel}
\usepackage{amsmath}
\usepackage{amsfonts}
\usepackage{amssymb}
\usepackage{graphicx}
\usepackage{lmodern}
\usepackage{hyperref}
\usepackage{xcolor}

\usepackage[left=2cm,right=2cm,top=2cm,bottom=2cm]{geometry}
\author{Tina Zwittnig 64200432}
\title{Poročilo 4. vaje pri predmetu OVS \\ Prikazovanje 3D slik v 2D}

\begin{document}
\maketitle
\pagebreak
\section{Butterworthova filtra}
\begin{verbatim}
function oMatrix = getFilterSpectrum(iMatrix, iD0, iType)

oMatrix = iMatrix;
[N, M] = size(iMatrix);

sredisce = [(N+1)/2, (M+1)/2];
if strcmp(iType, 'ILPF')
    for n = 1:N
        for m = 1:M
            D = norm([n,m]-sredisce);
            if D  <=iD0
                oMatrix(n,m) = 1;
            end
        end
    end
elseif strcmp(iType, 'BLPF')
    for n = 1:N
        for m = 1:M
            D = norm([n,m]-sredisce);
            oMatrix(n,m) = 1/(1+(D/iD0)^4);
        end
    end               
elseif strcmp(iType, 'IHPF')
    for n = 1:N
        for m = 1:M
            D = norm([n,m]-sredisce);
            if D  <=iD0
                oMatrix(n,m) = 1;
            end
        end
    end
    oMatrix = 1-oMatrix;
elseif strcmp(iType, 'BHPF')
    for n = 1:N
        for m = 1:M
            D = norm([n,m]-sredisce);
            oMatrix(n,m) = 1/(1+(iD0/D)^4);
        end
    end   
    
end
end
\end{verbatim}
\pagebreak
\section{Prikazi filtrov in filtriranih slik}
\begin{figure}[h!]
  \begin{center}
    \includegraphics[scale = 0.6]{img1.png}
    \caption{Originalna slika}
    \label{fig:}
  \end{center}
\end{figure}
\begin{figure}[h!]
  \begin{center}
    \includegraphics[scale = 0.2]{f1_16.png}
    \caption{Prikazani filtri in filtrirane slike, če najmanjšo dimenzijo množimo z 1/16}
    \label{fig:}
  \end{center}
\end{figure}
\pagebreak
\begin{figure}[h!]
  \begin{center}
    \includegraphics[scale = 0.2]{f1_4.png}
    \caption{Prikazani filtri in filtrirane slike, če najmanjšo dimenzijo množimo z 1/4}
    \label{fig:}
  \end{center}
\end{figure}
\begin{figure}[h!]
  \begin{center}
    \includegraphics[scale = 0.2]{f1_3.png}
    \caption{Prikazani filtri in filtrirane slike, če najmanjšo dimenzijo množimo z 1/3}
    \label{fig:}
  \end{center}
\end{figure}

Opazimo lahko, da je ima filter ILPF oz. IHPF oster rob, medtem ko imata filtra BHPF in BLPF zamegljen rob. Pri filtrirani sliki z ILPF filtrom lahko opazimo zvonenje (nekakšne vzporednice robovom). Medtem, ko ga (zvonenja) Butterworthov filter ne povzroča. 
Visoko prepustni filtri ostrijo oz. iščejo robove, medtem ko nizki prepustni filtri gladijo. 
\section{Enosmerna spektralna komponenta}
Sliko presilkamo s Four. transformacijo = G. Na preslikani sliki izračunamo amplitudni odziv, da dobimo $\lvert G \rvert$. Če iz amplitudnega odziva preberemo element na mestu $(n,m) =(0,0)$ oz. na indeksu $(1,1)$ dobimo spektralno komponento. Če jo delimo z $\sqrt{N\cdot M}$ dobimo ravno povprečno sivinsko vrednost, pri čimer $N, M$ predstavljata velikost slike.\\
To lahko tudi razberemo iz enačbe v navodilih za $G(m, n)$. Če je $n= m = 0$ je člen z eksponentom vedno 1, tako seštevamo ravno sivinske vrednosti slike. 

\end{document}
