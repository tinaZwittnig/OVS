\documentclass[12pt,a4paper]{article}
\usepackage[slovene]{babel}
\usepackage{amsmath}
\usepackage{amsfonts}
\usepackage{amssymb}
\usepackage{graphicx}
\usepackage{lmodern}
\usepackage{hyperref}
\usepackage{xcolor}
\usepackage{adjustbox}
\usepackage{multirow}
\usepackage[left=2cm,right=2cm,top=2cm,bottom=2cm]{geometry}
\author{Tina Zwittnig 64200432}
\title{Poročilo 6. vaje pri predmetu OVS \\ Geometrijske preslikave slik}


\begin{document}
\maketitle
\pagebreak
\section{Dodajanje interpolacije prvega reda}
\begin{verbatim}
function oImage = transformImage(iType, iImage, iDim, iP, iBgr, iOrder)
%transformImage transformira dano sliko z tipom transformacije. 
%vhodni parametri:
%   iType - tip tranforamcije 'affine' ali 'radial'
%   iImage - slika, podana v matrični obliki
%   iDim -  dimenzija pikslov podana v [dimX dimY]
%   iP - matrika dane transformacije
%   iBgr - barva odzadja
%   iOrder - željen red interpolacije 0 ali 1
%izhodni parametrri:
%   oImage - transformirana slika

%dimenzija slike
[Y, X] = size(iImage);
%inicializacija izhodne matrike
oImage = ones(Y,X) *iBgr;
for y = 0:Y-1
    for x = 0:X-1
        %koordinate trenutne točke
        pt = [x y].*iDim;
        %afina geometrijska preslikava
        if strcmp(iType, 'affine')
            pt = iP*[pt';1];
            pt = pt(1:2)';
        end
        if strcmp(iType, 'radial')
            U = getRadialValue(pt,iP{1})';
            pt = [U*iP{2}(:,1), U*iP{2}(:,2)];
        end
        pt = pt./iDim;
        %interpolacija reda 0
        if iOrder == 0 %princip najbližjega slikovnega elementa
            px = round(pt); 
            if px(1) <X &&px(2)<Y && px(1)>=0 && px(2)>=0% preverimo veljavnost lokacije slikovnega elementa
                oImage(y+1,x+1) = iImage(px(2)+1,px(1)+1);
        
            end
        elseif iOrder == 1 %interpolacija 1. reda
            px  = floor(pt);
            if px(1) < size(iImage,2) && px(2) < size(iImage,1) && 0<px(1) && 0< px(2)% preverimo veljavnost lokacije slikovnega elementa
                p1 = abs(pt(1)- px(1)) * abs(pt(2)-px(2));
                p2 = abs(pt(1) - (px(1)+1)) * abs(pt(2) -px(2));
                p3 = abs(pt(1) - px(1)) * abs(pt(2) - (px(2) +1));
                p4 = abs(pt(1) - (px(1)+1)) * abs(pt(2) - (px(2) +1));
              
                sa = iImage(px(2),px(1));
                sb = iImage(px(2),px(1)+1);
                sc = iImage(px(2)+1,px(1));
                sd = iImage(px(2)+1,px(1)+1);
                s = floor(p4 * sa + p3 * sb + p2 * sc +p1 * sd);
                oImage(y+1,x+1) = s;
            end
        end
        
    end
end
end
\end{verbatim}
\begin{figure}[h!]
  \begin{center}
    \includegraphics[scale = 0.6]{originalna.png}
    \caption{Originalna slika}
    \label{fig:}
  \end{center}
\end{figure}
\pagebreak
\begin{figure}[h!]
  \begin{center}
    \includegraphics[scale = 0.5]{prva_nictega.png}
    \caption{Slika, po rotaciji in striženju interpolirana z ničtim redom}

  \end{center}
\end{figure}

\begin{figure}[h!]
  \begin{center}
    \includegraphics[scale = 0.5]{prva_prvega.png}
    \caption{Slika, po rotaciji in striženju interpolirana z prvim redom}

  \end{center}
\end{figure}
\pagebreak
\section{Afine preslikave}
\begin{figure}[h!]
  \begin{center}
    \includegraphics[scale = 0.6]{druga_orig.png}
    \caption{Originalna slika}

  \end{center}
\end{figure}
\subsection{Skaliranje s parametri $k_x = 0.7$ in $k_y = 1.4$}
\begin{figure}[h!]
  \begin{center}
    \includegraphics[scale = 0.5]{druga_a.png}
    \caption{Skalirana slika}
  
  \end{center}
\end{figure}
\pagebreak
\subsection{translacija s parametri $t_x = 20\text{mm}$ in $ty = -30\text{mm}$}
\begin{figure}[h!]
  \begin{center}
    \includegraphics[scale = 0.4]{druga_b.png}
    \caption{Translirana slika}
  \end{center}
\end{figure}
\subsection{Rotacija z $\varphi = -30^\circ$}
\begin{figure}[h!]
  \begin{center}
    \includegraphics[scale = 0.4]{druga_c.png}
    \caption{Rotirana slika}
  \end{center}
\end{figure}
\pagebreak
\subsection{Strig s parametri $g_{xy} = 0.1$ in $g_{yx} = 0.5$}
\begin{figure}[h!]
  \begin{center}
    \includegraphics[scale = 0.4]{druga_d.png}
    \caption{Slika s strigom}

  \end{center}
\end{figure}
\subsection{ Translacija in rotacija $t_x = -10\text{mm}, t_y = 20\text{mm}, \varphi = 15^\circ$}
\begin{figure}[h!]
  \begin{center}
    \includegraphics[scale = 0.4]{druga_e.png}
    \caption{Rotirana in translirana slika}

  \end{center}
\end{figure}
\pagebreak
\subsection{Skaliranje, translacija in rotacija $k_x = k_y = 0.7, t_x = 30\text{mm}, t_y = -20\text{mm}, \varphi = -15^\circ$}

\begin{figure}[h!]
  \begin{center}
    \includegraphics[scale = 0.4]{druga_f.png}
    \caption{Rotirana in translirana in skalirana slika}

  \end{center}
\end{figure}

\section{Preslikavi}
Preslikava iz vprašanja 2e se imenuje Togi premik. Zanj je značilno, da ohranja razdalja. Torej razdalja originalnih točk je enaka razdalji preslikanih točk. $d(A,B) = d(A',B')$ Togi premik je injektivna preslikava (kar pomeni, da se dve različni točki ne moreta preslikati v isto.)
Preslikava iz vprašanja 2f pa je podobnostna transformacija. Za podobnostno transformacijo je značilno, da je izogonalna, kar pomeni, da ohranja pravokotnost koordinatnih osi. S tem se tudi ohranja velikost kotov in vzporednost. Dolžine in položaji točk v tej preslikavi pa se ne ohranjajo. 
Preslikava je podobnostana, če velja $k_x = k_y$
\pagebreak
\section{Preslikave z izhodiščem v sredini slike}
\begin{figure}[h!]
  \begin{center}
    \includegraphics[scale = 0.5]{tretja_a.png}
    \caption{Rotacija v središču slike}
  \end{center}
\end{figure}
\begin{figure}[h!]
  \begin{center}
    \includegraphics[scale = 0.5]{tretja_b.png}
    \caption{Strig v središču slike}
  \end{center}
\end{figure}
\pagebreak
\begin{verbatim}
phi = -30;
rotacijapremik = getParameters('affine',[1,1] , 
                                [-imsize(1)/2 *pxDim(1), -imsize(2)/2 *pxDim(2)], 
                                phi, [0 0]);
premik = getParameters('affine', [1,1], 
                        [imsize(1)/2*pxDim(1), imsize(2)/2*pxDim(2)],0,[0,0]);
RT= premik*rotacijapremik;
transf = transformImage('affine',slika , pxDim, inv(RT),63,1);
slika_rotirana = displayImage(transf, 'rotirana_sredina',gX, gY);
saveas(slika_rotirana, 'tretja_a.png')


gxyd = 0.1;
gyxd = 0.5;
strig_premik = getParameters('affine', [1 1], 
                            [-imsize(1)/2 *pxDim(1), -imsize(2)/2 *pxDim(2)], 
                            0, [gxyd, gyxd]);
RT2 = premik*strig_premik;
slika_strizenje = transformImage('affine',slika , pxDim, inv(RT2) ,63,1);
strig_slika = displayImage(slika_strizenje, 'Striženje slika',gX, gY);
saveas(strig_slika, 'tretja_b.png')
\end{verbatim}
\section{Radialna preslikava}
\begin{figure}[h!]
  \begin{center}
    \includegraphics[scale = 0.5]{peta_a.png}
    \caption{Originalna slika s kontrolnimi točkami}
  \end{center}
\end{figure}
\pagebreak
\begin{figure}[h!]
  \begin{center}
    \includegraphics[scale = 0.5]{peta_b.png}
    \caption{Preslikana slika s preslikanimi kontrolnimi točkami}
  \end{center}
\end{figure}


\pagebreak

\begin{figure}[h!]
  \begin{center}
    \includegraphics[scale = 0.5]{peta_a2.png}
    \caption{Originalna slika s kontrolnimi točkami}
    \label{fig:}
  \end{center}
\end{figure}
\begin{figure}[h!]
  \begin{center}
    \includegraphics[scale = 0.5]{peta_b2.png}
    \caption{Preslikana slika s preslikanimi kontrolnimi točkami}
    \label{fig:}
  \end{center}
\end{figure}
Vidimo, da preslikava ne deluje pravilno. Kotne točke, se ne spremenijo, ker so enake (torej se preslikajo same vase). Če opazujemo notranje točke, vidimo, da se preslikane kontrolne točke preslikajo v originalne kotrolne točke. Podobno kot pri afini preslikavi moramo tudi tukaj vzeti inverz radialne preslikave oz. slikati preslikane kontrolne točke v kontrolne točke, da dobimo prav rezultat. To se zgodi ker preslikava ne slika iz prostora indeksov v prostor indeksov. 

\begin{figure}[h!]
  \begin{center}
    \includegraphics[scale = 0.5]{peta_a3.png}
    \caption{Pravilno preslikana slika}
    \label{fig:}
  \end{center}
\end{figure}
\begin{figure}[h!]
  \begin{center}
    \includegraphics[scale = 0.5]{peta_b3.png}
    \caption{Pravilno preslikana slika}
    \label{fig:}
  \end{center}
\end{figure}

Kot vidimo preslikava sedaj deluje pravilno. Konrolne točke res preslika v preslikane kontrolne točke. 

\begin{thebibliography}{9}
\bibitem{MIP}\textbf{Togi premik}, Wikipedia, dostopno na: \url{https://sl.wikipedia.org/wiki/Togi_premik} [ogled 16.11.2020]
\bibitem{CT}\textbf{Podobnostna transformacija}, M. Kuhar, FGG, dostopno na: \url{http://fgg-web.fgg.uni-lj.si/~/mkuhar/Pouk/RSG/gradivo/Podobnostna_transformacija-gradivo.pdf} [ogled 16.11.2020]

\bibitem{FRI}\textbf{Geometrija in transformacije}, FRI, dostopno na: \url{https://ucilnica.fri.uni-lj.si/pluginfile.php/159572/mod_resource/content/0/12%20Geometrija%20in%20transformacije.pdf}[ogled 16.11.2020]

\end{thebibliography}
\end{document}
