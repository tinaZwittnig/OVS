\documentclass[12pt,a4paper]{article}
\usepackage[slovene]{babel}
\usepackage{amsmath}
\usepackage{amsfonts}
\usepackage{amssymb}
\usepackage{graphicx}
\usepackage{lmodern}
\usepackage{hyperref}
\usepackage{xcolor}

\usepackage[left=2cm,right=2cm,top=2cm,bottom=2cm]{geometry}
\author{Tina Zwittnig 64200432}
\title{Poročilo 11. vaje pri predmetu OVS \\ Vektor premika}

\begin{document}
\maketitle
\pagebreak
\section{napoved slike}
Izračunali smo vektorje premika in s pomočjo njih napovedali, kje se bo nahajal kvadrat v naslednjih framih. Vidimo, da razlika ni črna, to pa je pripisati velikosti bloka. Če bi vezli manjši blok, bi bila napoved verjetno boljša.
\begin{figure}[h!]
  \begin{center}
    \includegraphics[scale = 0.4]{prva.png}
    \caption{Izris slik, napovedane slike in slika razlike med napovedano sliko in sliko, katero smo poskušali napovedati}
    \label{fig:}
  \end{center}
\end{figure}
\pagebreak
\section{Video}
\begin{verbatim}
  ovid = VideoWriter('VektorjiPremika.avi');

ivid = VideoReader('simple-video.avi');
fram = ivid.Duration * ivid.FrameRate;
open(ovid);

for i = 1:fram-1

    I1 = read(ivid, i);
    I2 = read(ivid, i+1);

    bSize = [8,8];
    sSize = 2^4-1;

    [MF, CP] = blockMatching(I1, I2, bSize, sSize);
    quiver(CP(:,:,1), CP(:,:,2), MF(:,:,1), MF(:,:,2), 'r-') %ne uporabimo funkcije, da se nam ne odpre 150 figure.
    set(gca, 'Ydir', 'reverse');
    axis image
    frame = getframe(gcf);
    writeVideo(ovid,frame);
end

close(ovid);
\end{verbatim}
\pagebreak
\section{polje  vektorjev  premika}
\begin{figure}[h!]
  \begin{center}
    \includegraphics[scale = 0.4]{tretja.png}
    \caption{Originalna slika, slika pri naslednjem frame-u, vektorji premika in Superponirani vektorji premika}
    \label{fig:}
  \end{center}
\end{figure}

Pri realnem videu vidimo, da je vektorjev premika več, in da so bolj naključno razporejeni, kot pri prvem videu. 

\end{document}
